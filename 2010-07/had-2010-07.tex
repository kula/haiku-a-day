% Copyright (C) 2010 Thomas L. Kula
% All Rights Reserved
%
% See the file LICENSE for license terms.
\documentclass[12pt]{article}
\usepackage{graphicx}
\usepackage{rotating}
\usepackage{fix-cm}
\setlength{\paperwidth}{5.5in}
\setlength{\paperheight}{8.5in}
\setlength{\textheight}{7.45in}
\setlength{\topmargin}{-1.0in}
\setlength{\oddsidemargin}{-0.5in}
\setlength{\evensidemargin}{-0.5in}
\setlength{\textwidth}{4.0in}
\setlength{\parindent}{0in}
\setlength{\parskip}{3mm}
\usepackage[print]{booklet} \nofiles
\source{\magstep0}{5.5in}{8.5in}
\target{\magstep0}{11in}{8.5in}
\setpdftargetpages
\pagestyle{empty}
\begin{document}


\begin{center}
{\fontsize{36}{48}\selectfont \textsc{Haiku a Day }} \\[0.5cm]
{\bfseries End of the First Glorious Five Year Plan}
\end{center}

\vspace*{3.5cm}

{\fontsize{20}{40}\selectfont 

Every day it's

Just seventeen syllables

But that is enough

}

\vspace*{5.0cm}
\begin{center}
{\large{Issue 61: July 2010}} \\[5mm]
{\fontsize{8}{8}\selectfont  \textsc{ St. Joshua Norton Press }} \\[1mm]
{\fontsize{6}{6}\selectfont Mathom House in Midtown \textbar The People's Republic of Ames }
\end{center}


\newpage

Five years ago I was sitting in my duplex in Ames, watching a
printer I don't use any more spit out the first issue of Haiku
a Day. It was my first real experiment with creating a zine,
so there were plenty of false starts, the least of which came
from a printer that didn't do duplex so I had to do the mental
gymnastics to flip the paper around {\em just right} so it would
print out properly. 

It's much easier now, although once a year, on the anniversary,
I try to do something special just to keep things interesting.
I hope you enjoy this issue, and I look forward to making
many more.

--- Thomas

http://kula.tproa.net/had/ |  kula@tproa.net

Download this and previous HADs at the website, so you can
print out your own (DIY, yeah!) or if you want me to send
you one, send me your address, and maybe a stamp if you
are feeling nice. Or send me something you've made ---
trades always appreciated, postcards are nice too.


1 July 2010

O Fair Canada \\
I yearn to return to you \\
Apart far too long

2 July 2010

The day, smoothly goes \\
Slams to a frustrating stop \\
A long weekend saves

3 July 2010

Beside the Huron \\
The Night of the Hunter plays \\
Screen glows in the night

\newpage

4 July 2010

Too hot to do much \\
I spend Independence Day \\
In where it is cool

5 July 2010

A day off is filled \\
Scrambling to finish errands \\
No rest here today

6 July 2010

A plan in my mind \\
Changes to one different \\
At the hardware store

7 July 2010

A bit of crafty \\
Producing a pleasant glow \\
Inside of my mind

8 July 2010

The printer now done \\
Can only mean one thing left \\
Staplepalooza!

9 July 2010

In these boxes lie \\
Everything you might need \\
Plus a bunch of tape

10 July 2010

Those hours standing \\
Thousands of people go by \\
Sitting well with me

\newpage

11 July 2010

Why am I up now? \\
The lure of sleeping in strong \\
But not strong enough

12 July 2010

Like a waterfall \\
Just one that can catch fire \\
Fuel leak in my car

13 July 2010

It can be cool here \\
If you sit in the shade and \\
Don't move a muscle

14 July 2010

In an asphalt sea \\
An oasis of green lives \\
Against all reason

15 July 2010

The drone of a fan \\
A one-chord symphony plays \\
Eat your heart out, Cage

16 July 2010

A box holds a slot \\
The slot, taking envelopes, \\
Sends them on their way

17 July 2010

What we used to do \\
In keeping fire at bay \\
Fills a museum

\newpage

18 July 2010

Busy bees buzzing \\
On flowers of all the hues \\
Grey skies sit above

19 July 2010

Once more 'round the Sun \\
Thirty-two times in my life \\
I'm getting dizzy

20 July 2010

Bluish-grey billows \\
Massing above a peach sky \\
Rain --- but colorful

21 July 2010

Lazyness tonight \\
Chinese takeout tempting me \\
Spring roll's siren song

22 July 2010

Air cools, but dances \\
To the drum beats of thunder \\
Inside it grows dark

23 July 2010

It's balls ass hot outside \\
I breathe out and it feels cool \\
That's just not right, folks

24 July 2010

Long into the night \\
I wait for power's return \\
Generators sad

\newpage

25 July 2010

Dare I venture out? \\
Will I become sticky goo? \\
Hey, it's nice outside

26 July 2010

Venture from the cave \\
A life, somewhat more normal \\
Eyes bleary, blinking

27 July 2010

Things that can be done \\
Stopped by the lack of magic \\
Bits drive me crazy

28 July 2010

Between two large fields \\
A slender slip dividing \\
Limiting movement

29 July 2010

What once appeared new \\
Fading over time, dulling, \\
Becomes sad, yet proud

30 July 2010

Glorious day off \\
Wandering around, lazy \\
My mind is relaxed

31 July 2010

Where does this come from? \\
I just cleaned this thing last week \\
It's dirty again

\newpage

\begin{center}
\bf{The Story of Haiku a Day}
\end{center}

The real start of Haiku a Day started 8-and-a-half years
before the first issue every came out. The second semeter
of my freshman year of college I was introduced to the
first of the large mailing lists my friends and I were on ---
a supremely wonderful explosion of madness and weirdness,
designed to exquisitely waste large amounts of time before
things like Facebook made that much more efficient to do  ---
the follow-on of which I am still on today. It became our
habit on occasion to have large conversations entirely in
Haiku.

A pause here for purists: what we used, and what Haiku a Day
has always limited itself to, is the rather narrow view of
a haiku as something with the 5-7-5 syllable pattern. The 
traditional Japanese poetic form of the haiku has much more
convention than that, and those who are good at it produce
sublimely wonderful works of art. But for this, I am more
intested simply in the challenge of trying to convey a 
thought just constrained to 5-7-5.

The pragmatic start of Haiku a Day was a trip I took to
Pittsburgh in 2005. I had discovered Copacetic Comics ---
and if you ever find yourself in Squirrel Hill, do
yourself a favor and find Copacetic, which is tiny and
out of the way but has quite possibly the highest concentration
of awesome I've ever encountered --- and there I picked up
one of the Snakepit anthologies. Ben Snakepit, for years,
has documented every day of his life with a simple, three
pane comic. I was stuck by this idea, and with the idea of
haiku in my mind, resolved to do something like that.
Seventeen syllables a day, every day. And thus it was
born.

I hope you enjoy it as much as I enjoy sending it out.

\newpage

\thispagestyle{empty}
\vspace*{14cm}
\begin{sideways}
\Large{Thomas L. Kula}
\end{sideways}
\begin{sideways}
\Large{PO Box 980461}
\end{sideways}
\begin{sideways}
\Large{Ypsilanti MI 48198}
\end{sideways}


\end{document}


